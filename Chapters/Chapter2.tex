

\chapter{Introducción específica} % Main chapter title

\label{Chapter2}

%----------------------------------------------------------------------------------------
%	SECTION 1
%----------------------------------------------------------------------------------------
Esta sección presenta una breve introducción técnica a las herramientas hardware y software utilizadas en el trabajo.

\section{Tecnologías de hardware utilizadas}

\subsection{Espressif ESP32}


ESP32 es una serie de microcontroladores de bajo costo y bajo consumo creado y desarrollado por \textit{Espressif Systems} embebido en un chip con Wi-Fi integrado (2.4 GHz band) y Bluetooth de modo dual. Emplea dos cores Xtensa® 32-bit LX6 CPU. Incluye interruptores de antena integrados, balun de RF, amplificador de potencia, amplificador de recepción de bajo ruido, filtros, un co-procesador ULP (Ultra Low Power),  módulos de administración de energía y varios periféricos.
En la siguiente imagen (\ref{fig:esp32}) se puede apreciar la placa ESP32-WROOM-32D utilizada para el desarrollo del presente trabajo.

\begin{figure}[h]
    \centering
    \includegraphics[scale=0.25]{esp32}
    \caption{Microcontrolador ESP32-WROOM-32D.}
    \label{fig:esp32}
\end{figure}


\subsection{Sensor de temperatura y humedad DHT11}

El DHT11 es un sensor digital de temperatura y humedad relativa de bajo costo y fácil uso. Integra un sensor capacitivo de humedad y un termistor para medir el aire circundante, y muestra los datos mediante una señal digital en el pin de datos (no posee salida analógica). Utilizado en aplicaciones académicas relacionadas al control automático de temperatura, aire acondicionado, monitoreo ambiental en agricultura y más. En la figura \ref{fig:dht11} se puede apreciar una imagen del mismo.

\begin{figure}[h]
    \centering
    \includegraphics[scale=0.25]{dht11}
    \caption{Sensor DHT11.}
    \label{fig:dht11}
\end{figure}

\subsection{Sensor de presión BMP280}

El BMP280 es un sensor de presión barométrica absoluta, especialmente factible para aplicaciones móviles que puede ser utilizado con I2C o SPI. Permite alta precisión y linealidad, estabilidad a largo plazo, alta robustez a un muy bajo consumo. En la figura \ref{fig:bmp280} se puede apreciar una imagen del mismo.

\begin{figure}[h]
    \centering
    \includegraphics[scale=0.25]{bmp280}
    \caption{Sensor BMP280.}
    \label{fig:bmp280}
\end{figure}

\subsection{Fotoresistor como sensor de luminosidad}

El fotoresistor es una resistencia eléctrica que varía su valor en función de la cantidad de luz que incide sobre su superficie. 
Cuando el fotoresistor no está expuesto a radiaciones luminosas, los electrones están firmemente unidos en los átomos que lo conforman, por lo que alcanza su máxima resistencia eléctrica, y cuando sobre él inciden radiaciones luminosas, esta energía libera electrones con lo cual el material se hace más conductor, y se disminuye su resistencia. En la figura \ref{fig:fotoresistor} se puede apreciar una imagen del mismo.

\begin{figure}[h]
    \centering
    \includegraphics[scale=0.2]{fotoresistor}
    \caption{Fotoresistor.}
    \label{fig:fotoresistor}
\end{figure}

\subsection{Joystick analógico}

El módulo de joystick analógico está construido sobre el montaje de dos potenciómetros en un ángulo de 90 grados. Los potenciómetros están conectados a una palanca corta centrada por resortes. Este módulo produce una salida de alrededor de 2,5 V cuando inicialmente se encuentra en posición de reposo (en el centro), mientras que en el trayecto del desplazamiento de la palanca hará que la salida varíe de 0V a 5V
dependiendo de su dirección X e Y. Al conectar este módulo a un microcontrolador se puede leer un valor de alrededor de 512 en su posición de reposo mientras que al moverlo cambia entre 0 y 1023 dependiendo de su posición. En la figura \ref{fig:joystick} se puede apreciar una imagen del mismo.

\begin{figure}[h]
    \centering
    \includegraphics[scale=0.10]{joystick}
    \caption{Joystick analógico.}
    \label{fig:joystick}
\end{figure}


\subsection{Display LCM1602A}
El display LCM1602A consta de una pantalla de cristal líquido de 1602 caracteres, en un módulo de matriz de puntos para mostrar letras, números y caracteres, etc. Permite representar dos filas con hasta 16 caracteres en cada una y dado que se encuentra integrado a una interfaz adaptadroa I2C puede ser controlado por este protocolo. En la figura \ref{fig:display} se puede apreciar una imagen del mismo.

\begin{figure}[h]
    \centering
    \includegraphics[scale=0.10]{display}
    \caption{Display LCM1602A.}
    \label{fig:display}
\end{figure}

\subsection{Motores de corriente continua}
El motor DC (o corriente continua), pertenece a la clase de los electromotores y sirve principalmente para transformar la energía eléctrica en energía mecánica. Estos motores operan con un voltaje entre 3 y 6 Volts, corriente de 150 mA, permiten una velocidad de entre 90 y 200 RPM y un torque de entre 0,15 Nm y 0,60 Nm. En la figura \ref{fig:dc_motors} se puede apreciar una imagen del mismo.


%\begin{figure}[htbp]
%\centering
\begin{center}
  \includegraphics[scale=0.3]{dc_motors}
    \captionof{figure}{Motor de corriente continua.}
    \label{fig:dc_motors}
\end{center}
  
%\end{figure}



\section{Tecnologías de software utilizadas} 

\subsection{Marco de trabajo ESP-IDF}

Espressif proporciona recursos básicos de hardware y software para ayudar a los desarrolladores de aplicaciones a realizar sus ideas utilizando el hardware de la serie ESP32. El framework de software de Espressif está destinado al desarrollo de aplicaciones de Internet de las cosas (IoT) con Wi-Fi, Bluetooth, administración de energía y varias otras características del sistema.
Sus componentes son:
\begin{enumerate}
	\item Toolchain para compilar el codigo para ESP32,
	\item Build tools - con utilidades como CMake y Ninja para construir la aplicación completa para ESP32,
	\item ESP-IDF que esencialmente contiene la API de desarrollo (software base y bibliotecas complementarias) para ESP32 y scripts para ejecutar Toolchain.
	
\end{enumerate}
En la figura \ref{fig:esp-idf} se puede apreciar una imagen del proceso de desarrollo y despliegue usando el framework ESP-IDF.

\begin{figure}[h]
    \centering
    \includegraphics[scale=0.5]{esp-idf}
    \caption{Proceso de desarrollo utilizando ESP-IDF.}
    \label{fig:esp-idf}
\end{figure}

%\begin{center}
%\end{center}
%\includegraphics[scale=0.25]{espressif}

\subsection{Plataforma Docker}

Docker es un proyecto de código abierto que automatiza el despliegue de aplicaciones dentro de contenedores de software, proporcionando una capa adicional de abstracción y automatización de virtualización de aplicaciones en múltiples sistemas operativos.​ Docker utiliza características de aislamiento de recursos del kernel Linux, tales como cgroups y espacios de nombres (namespaces) para permitir que contenedores livianos independientes se ejecuten en paralelo de manera aislada evitando la sobrecarga de iniciar y mantener máquinas virtuales.

%\includegraphics[scale=0.15]{docker}

\subsection{Visual Studio Code}

Visual Studio Code es un editor de código fuente desarrollado por Microsoft para Windows, Linux, macOS y Web. Incluye soporte para la depuración, control integrado de Git, resaltado de sintaxis, finalización inteligente de código, fragmentos y refactorización de código. 

%\includegraphics[scale=0.15]{vscode}

\subsection{Sistema operativo Ubuntu}
Ubuntu es una distribución Linux basada en Debian GNU/Linux y patrocinado por Canonical, que incluye principalmente software libre y de código abierto. Puede utilizarse en ordenadores y servidores, está orientado al usuario promedio, con un fuerte enfoque en la facilidad de uso y en mejorar la experiencia del usuario. 

%\includegraphics[scale=0.25]{ubuntu}
