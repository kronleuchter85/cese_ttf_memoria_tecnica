% Chapter Template

\chapter{Conclusiones} % Main chapter title

\label{Chapter5} % Change X to a consecutive number; for referencing this chapter elsewhere, use \ref{ChapterX}


%----------------------------------------------------------------------------------------

%----------------------------------------------------------------------------------------
%	SECTION 1
%----------------------------------------------------------------------------------------

% \begin{itemize}\item ¿Cuál es el grado de cumplimiento de los requerimientos?
En el presente trabajo se ha implementado un robot de exploración ambiental que cumple con todos los requerimientos establecidos en el plan de proyecto. Los requerimientos funcionales han sido abordados mediante la implementación de los módulos de desplazamiento, control de movimiento, medición de parámetros ambientales y visualización, explicados en las secciones 3 y 4. Los requerimientos de documentación han sido cubiertos con los respectivos documentos referenciados en la sección 4. Los requerimientos de testing se cumplieron a través de tests unitarios midiendo el nivel de cobertura, además de la verificación y validación funcional de cada módulo (\textit{smoke test}) explicada en la sección 4. Los requerimientos de interfaz han sido completados como parte de la implementación del módulo de visualización explicados en la sección 3 y evidenciados en la sección 4.
Finalmente, de los requerimientos opcionales se implementó la comunicación inalámbrica entre el joystick y el robot mediante UDP sobre TCP/IP.

% \item ¿Cuán fielmente se puedo seguir la planificación original (cronograma incluido)?
Con respecto a la planificación original \cite{Robot_Planificacion}, durante la implementación del trabajo se produjeron eventos que afectaron los supuestos sobre la capacidad del alumno, lo que generó el riesgo de demora en la entrega. Este riesgo, contemplado en el plan del proyecto, fue aceptado (no mitigado) para no sacrificar ni el alcance ni la calidad, lo que resultó en un retraso en el plan original.


% \item ¿Se manifestó algunos de los riesgos identificados en la planificación? ¿Fue efectivo el plan de mitigación? ¿Se debió aplicar alguna otra acción no contemplada previamente?
Además del riego de demora, se presentó un desvío en los costos, generado por la subestimación de ciertos componentes adicionales, como por ejemplo, los módulos L298N, las baterías recargables AA y además de las plaquetas de montaje. Se logró mitigar con las acciones establecidas en el plan original, utilizando el presupuesto reservado como Varios/Imprevistos. Resultó especialmente útil estimar el presupuesto en dólares estadounidenses.

% \item Si se debieron hacer modificaciones a lo planificado ¿Cuáles fueron las causas y los efectos?

Fuera de lo mencionado en cuanto a desvío en tiempo y costos, no hubo modificaciones en cuanto al alcance ni calidad esperada. Además, se logró el cumplimiento de uno de los requerimientos adicionales: implementación del desarrollo como parte de un ciclo de integración continua usando productos de Google Cloud Platform. También se logró la cuantificación del nivel de cobertura de código de los test unitarios y se elaboró una documentación exhaustiva que incluye dos listas de reproducción de videos en Youtube para la construcción y demostración del producto.

% \item ¿Qué técnicas resultan útiles para el desarrollo del proyecto y cuáles no tanto?

Durante la implementación del trabajo, fueron utilizadas innumerables técnicas y conocimientos adquiridos en la Carrera de Especialización de Sistemas Embebidos, incluyendo conceptos como: prototipado de circuitos en protoboard; diseño, construcción y modularización de plaquetas integradas, protocolos utilizados en sistemas embebidos, modularización de componentes y servicios en FreeRTOS, desarrollo de firmware utilizando el SDK Espressif ESP-IDF, y la implementación de test unitarios con Ceedling y CUnit en sistemas embebidos, entre otros.



% \end{itemize}


%----------------------------------------------------------------------------------------
%	SECTION 2
%----------------------------------------------------------------------------------------
\section{Próximos pasos}

Concluída la implementación del sistema embebido del robot de exploración ambiental planteado, se propone como siguiente paso la implementación en un caso de uso IoT de robot de exploración de datos ambientales críticos, en el que se debe integrar el presente sistema embebido con un sistema backend en la nube Además, por motivos de inmutabilidad y auditoria, ciertos datos deberán poder persistir en una red blockchain. En el siguiente enlace se puede apreciar el plan de proyecto \cite{Robot_CEIOT_Planificacion_doc}.