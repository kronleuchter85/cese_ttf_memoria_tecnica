% Chapter Template

\chapter{Conclusiones} % Main chapter title

\label{Chapter5} % Change X to a consecutive number; for referencing this chapter elsewhere, use \ref{ChapterX}


%----------------------------------------------------------------------------------------

%----------------------------------------------------------------------------------------
%	SECTION 1
%----------------------------------------------------------------------------------------

% \begin{itemize}\item ¿Cuál es el grado de cumplimiento de los requerimientos?
En el presente trabajo se ha implementado un robot de exploración ambiental cumpliendo con todos los requerimientos establecidos en el plan de proyecto. Los requerimientos funcionales han sido abordados mediante la implementación de los módulos de desplazamiento, control de movimiento, medición de operaciones ambientales y visualización, explicados en las secciones 3 y 4. Los requerimientos de documentación han sido cubiertos con los respectivos documentos referenciados en la sección 4. Los requerimientos de testing han sido completados realizando tests unitarios midiendo el nivel de cobertura, además de la verificación y validación funcional de cada módulo (smoke test) explicada en la sección 4. Los requerimientos de interfaz han sido completados como parte de la implementación del módulo de visualización explicados en la sección 3 y evidenciado en la sección 4.
Finalmente, de los requerimientos opcionales se implementó la comunicación inalámbrica entre el joystick y el robot mediante UDP sobre TCP/IP.

% \item ¿Cuán fielmente se puedo seguir la planificación original (cronograma incluido)?
Con respecto a la planificación original, durante la implementación del trabajo se produjeron eventos que hicieron que los supuestos vinculados a la capacidad del alumno no se mantuvieran, y se manifestó el riesgo de demora en la entrega, contemplado en la planificación del proyecto, que al ser aceptado (no mitigado) con el fin de no sacrificar alcance ni calidad, generó una demora en el plan original.


% \item ¿Se manifestó algunos de los riesgos identificados en la planificación? ¿Fue efectivo el plan de mitigación? ¿Se debió aplicar alguna otra acción no contemplada previamente?
Además del riego de demora, se manifestó el riesgo de desvio en costos,y se produjo como consecuencia de haber desestimado la necesidad de ciertos componentes adicionales, como por ejemplo, los módulos L298N y las baterías recargables AA, además de las plaquetas de montaje. Se logró mitigar con las acciones establecidas en el plan original, utilizando el presupuesto reservado como Varios / Imprevistos y fue de mucha utilidad estimar el presupuesto en dólares estadounidenses.

% \item Si se debieron hacer modificaciones a lo planificado ¿Cuáles fueron las causas y los efectos?

Fuera de lo mencionado en cuanto a desvío en tiempo y costos no hubo modificaciones en cuanto al alcance ni calidad esperada, siendo posible además, el cumplimiento de uno de los requerimientos adicionales, la implementación del desarrollo como parte de un ciclo de integración continua usando productos de Google Cloud Platform, la cuantificación del nivel de cobertura de código de los test unitarios y una documentación exhaustiva incluyendo dos listas de reproducción de videos en Youtube para la construcción y demostración del producto.

% \item ¿Qué técnicas resultan útiles para el desarrollo del proyecto y cuáles no tanto?

Durante la implementación del proyecto fueron utilizadas innumerables técnicas y conocimientos adquiridos en la Carrera de Especialización de Sistemas Embebidos, incluyendo conceptos de: prototipado de circuitos en protoboard; diseño, construcción y modularización de plaquetas integradas; protocolos utilizados en sistemas embebidos; modularización de componentes y servicios en FreeRTOS; desarrollo de firmware utilizando el SDK Espressif ESP-IDF; y la implementación de test unitarios con Ceedling y CUnit en sistemas embebidos, entre otros.



% \end{itemize}


%----------------------------------------------------------------------------------------
%	SECTION 2
%----------------------------------------------------------------------------------------
\section{Próximos pasos}

Habiendo concluido con la implementación del sistema embebido del robot de exploración ambiental planteado, se propone como siguiente paso la implementación en un caso de uso IoT de robot de exploración de datos ambientales criticos, en el cual se debe integrar el presente sistema embebido con un sistema backend en la nube, y adeamás, por motivos de inmutabilidad y auditoria debe poder persistir ciertos datos en una red blockchain. En el siguiente enlace se puede apreciar el plan de proyecto \cite{Robot_CEIOT_Planificacion_doc}.