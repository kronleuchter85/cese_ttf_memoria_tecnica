% Chapter 1

\chapter{Introducción general} % Main chapter title

\label{Chapter1} % For referencing the chapter elsewhere, use \ref{Chapter1} 
\label{IntroGeneral}

%----------------------------------------------------------------------------------------

% Define some commands to keep the formatting separated from the content 
\newcommand{\keyword}[1]{\textbf{#1}}
\newcommand{\tabhead}[1]{\textbf{#1}}
\newcommand{\code}[1]{\texttt{#1}}
\newcommand{\file}[1]{\texttt{\bfseries#1}}
\newcommand{\option}[1]{\texttt{\itshape#1}}
\newcommand{\grados}{$^{\circ}$}

%----------------------------------------------------------------------------------------

%\section{Introducción}

%----------------------------------------------------------------------------------------
\section{Motivación}

El presente proyecto es un emprendimiento personal que busca desarrollar un dispositivo robótico de exploración ambiental controlable a distancia con las funciones básicas de desplazamiento, medición y reporte de parámetros ambientales tales como presión, temperatura, humedad y luminosidad. 


%----------------------------------------------------------------------------------------

\section{Alcance y objetivos}

..

%----------------------------------------------------------------------------------------

\section{Estado del arte}

Los robots exploradores son dispositivos robotizados capaces de moverse de forma autónoma, y/o controlados a distancia, que han sido creados con el fin de reconocer y explorar un lugar o entorno donde una persona no pueda o deba acceder ya sea por motivos de capacidad, practicidad o seguridad. Por este motivo, en función de las necesidades de desplazamiento, existen diferentes sistemas de motricidad, como son por ejemplo, los bípedos, cuadrúpedos, con ruedas, tracción oruga, acuáticos/sumergibles, aéreos, etc. En cuanto a la forma de control, hay manejados por control remoto cableado o inalámbrico, habiendo equipos más sofisticados que gracias a aplicaciones de Inteligencia Artificial están preparados para desplazarse y tomar decisiones de forma autónoma. Algunos de los tipos de robots exploradores más conocidos son los espaciales, de minas, de rescate en catástrofes, de tuberías, acuáticos y/o submarinos, y de suelos.
