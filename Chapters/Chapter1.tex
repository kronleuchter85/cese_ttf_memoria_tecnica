% Chapter 1

\chapter{Introducción general} % Main chapter title

\label{Chapter1} % For referencing the chapter elsewhere, use \ref{Chapter1} 
\label{IntroGeneral}

%----------------------------------------------------------------------------------------

% Define some commands to keep the formatting separated from the content 
\newcommand{\keyword}[1]{\textbf{#1}}
\newcommand{\tabhead}[1]{\textbf{#1}}
\newcommand{\code}[1]{\texttt{#1}}
\newcommand{\file}[1]{\texttt{\bfseries#1}}
\newcommand{\option}[1]{\texttt{\itshape#1}}
\newcommand{\grados}{$^{\circ}$}

%----------------------------------------------------------------------------------------

%\section{Introducción}
Esta sección presenta la motivación, alcance, objetivos y requerimientos del producto en el marco del estado del arte y su importancia en la industria.  

%----------------------------------------------------------------------------------------
\section{Motivación}

La motivación del presente trabajo fue primeramente volcar y unificar en un emprendimiento personal los conceptos aprendidos en la especialización de Sistemas Embebidos, con una arquitectura robusta que pueda ser extrapolada a otros casos de uso de valor en la industria como por ejemplo la exploración de suelos en el agro, la exploración submarina para la perforación de pozos de petróleo, o los mencionados más adelante en el estado del arte.
Por otra parte, se buscó desarrollar un producto que pueda contribuir a aumentar la oferta de dispositivos robóticos exploradores en Argentina.

\section{Estado del arte}

Los robots exploradores son dispositivos robotizados capaces de moverse de forma autónoma, y/o controlados a distancia, que han sido creados con el fin de reconocer y explorar un lugar o entorno donde una persona no pueda o deba acceder ya sea por motivos de capacidad, practicidad o seguridad. Por este motivo, en función de las necesidades de desplazamiento, existen diferentes sistemas de motricidad, como son por ejemplo, los bípedos, cuadrúpedos, con ruedas, tracción oruga, acuáticos/sumergibles, aéreos, etc. En cuanto a la forma de control, los hay manejados por control remoto cableado o inalámbrico, habiendo equipos más sofisticados, que gracias a aplicaciones de Inteligencia Artificial, están preparados para desplazarse y tomar decisiones de forma autónoma. Algunos de los tipos de robots exploradores más conocidos son los espaciales, de minas, de rescate en catástrofes, de tuberías, submarinos, y de suelos.

Tanto en el ámbito académico como en la industria existen trabajos, proyectos, e implementaciones comerciales similares al presente trabajo, como por ejemplo: 

\begin{itemize}
	\item El prototipo robótico de exploración minera publicado en varios artículos \cite{latam-mining-robot-minero-unsj}, \cite{diario-de-cuyo-prototipo-robotico}, e impulsado por el Instituto de Automática de la Facultad de Ingeniería de la Universidad Nacional de San Juan en el marco de un convenio con la Comisión Nacional de Energía Atómica y el Gobierno argentino \cite{comunicacion-unsj-prototipo-convenio}.

	\item El robot de exploración terrestre denominado Geobot \cite{geobot} desarrollado por los ingenieros Nelson Dario García Hurtado y Melvin Andrés González Pino, de la universidad de Pamplona, capaz de realizar reconocimiento de zonas y manipulación de muestras de manera autónoma o asistida.

	\item El robot minero MIN-SIS 1.0 SDG-STR \cite{min-sis} desarrollado por los ingenieros Hernán L. Helguero Velásquez y Rubén Medinaceli Tórrez de la Universidad Técnica de Oruro, capaz de detectar gases, almacenar datos locales y enviar video e imágenes al puesto de mando.

	\item Spot \cite{spot}, desarrollado por Boston Dynamics, un robot explorador cuadrupedo de propósito general capaz de explorar, almacenar y enviar información en tiempo real.
	  
	\item BIKE \cite{bike_inspection}, desarrollado por Waygate Technologies, un robot con ruedas magnéticas, muy utilizado en la industria de petróleo y gas entre otras, capaz de desplazarse por el interior de tuberías para poder realizar inspecciones y comunicar hallazgos.

\end{itemize}






%----------------------------------------------------------------------------------------

\section{Alcance y objetivos}
A continuación se detallan las funcionalidades incluidas en el alcance del trabajo.

\begin{itemize}
	\item Sistema de desplazamiento terrestre.
	\item Operaciones de exploración
	\begin{itemize}	
		 \item Medición de humedad ambiental.
		 \item Medición de temperatura ambiental.
		 \item Medición de presión ambiental.
		 \item Medición de luminosidad ambiental.
	\end{itemize}

	\item Visualización de estado de exploración (lecturas de los sensores).
	\item Sistema de control por medio de un joystick cableado.
\end{itemize}

Queda fuera del alcance:
\begin{itemize}
	\item Locomoción por cualquier otro medio que no sea terrestre.
	\item Cualquier otra función no contemplada en este alcance.
\end{itemize}

%----------------------------------------------------------------------------------------



\section{Requerimientos}
A continuación se listan los requerimientos del producto:
\begin{enumerate}	
	\item Requerimientos funcionales		
	\begin{enumerate}			
		\item El sistema debe contar con funciones de desplazamiento para poder moverse hacia adelante y atrás, y poder girar radialmente hasta un ángulo de 360 grados.			
		\item El sistema debe ser capaz de realizar las siguientes operaciones de exploración:			
			\begin{enumerate}				
				\item medición de humedad ambiental,				
				\item medición de temperatura ambiental,				
				\item medición de luminosidad ambiental,				
				\item medición de presión ambiental.			
			\end{enumerate}			
		\item El sistema debe poder ser controlado a distancia mediante un joystick para que el dispositivo pueda realizar sus movimientos. En caso de que alguna de sus operaciones de exploración requiera algún mecanismo de control, el mismo también será integrado en el joystick.		
		\item El sistema debe proveer un mecanismo de visualización de las operaciones de exploración al usuario que controla el dispositivo para poder ver el estado y lectura de las operaciones de exploración.		
		\end{enumerate}	
	\item Requerimientos de documentación		
		\begin{enumerate}			
			\item documentación de arquitectura técnica a alto nivel del diseño del sistema.			
			\item documentación técnica de la implementación del software.			
			\item documentación técnica de la implementación del hardware.			
			\item manual de usuario.	
			\item informe de avance.
			\item memoria final.	
		\end{enumerate}	
	\item Requerimiento de testing		
		\begin{enumerate}			
			\item se debe incluir tests de integración de componentes,
			\item se debe incluir tests funcionales (smoke test) del producto general.		
		\end{enumerate}	
	\item Requerimientos de la interfaz		
		\begin{enumerate}			
			\item la interfaz de usuario debe permitir visualizar las lecturas de cada uno de los sensores,			
			\item debe haber una pequeña leyenda de la magnitud que se está midiendo y la unidad utilizada junto con el valor.		
		\end{enumerate}	
	\item Requerimientos opcionales		
		\begin{enumerate}			
			\item De interfaz: se permite agregar cualquier otra interfaz adicional que agregue mejoras en la experiencia de usuario			
			\item De operaciones de exploración: se permite agregar cualquier otra operación adicional de exploración que agregue valor a exploración.	
			\item De comunicación: se permite agregar comunicación inalámbrica.		
	\end{enumerate}
\end{enumerate}

