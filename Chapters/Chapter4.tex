% Chapter Template

\chapter{Ensayos y resultados} % Main chapter title

\label{Chapter4} % Change X to a consecutive number; for referencing this chapter elsewhere, use \ref{ChapterX}

%----------------------------------------------------------------------------------------
%	SECTION 1
%----------------------------------------------------------------------------------------
Esta sección presenta los diferentes prototipos realizados para determinar la viabilidad de cada una de las funcionalidades provistas, la metodología de desarrollo, testing, y finalmente los entregables finales del trabajo.

\section{Proceso de desarrollo y aseguramiento de calidad}
\label{sec:pruebasHW}

Para el proceso de desarrollo se realizaron pruebas de concepto de las diferentes funcionalidades utilizando como materiales la bibliografía encontrada en internet, las hojas de datos y los ejemplos de código provistos por el SDK y librerías empleadas. Una vez logrado el objetivo funcional de componente se optimizó y encapsuló cada módulo para ser integrado de manera individual a un prototipo integrador sin afectar el funcionamiento de cualquier otro módulo.
De esta manera se desarrolló un prototipo integrador como la sumatoria de todos los módulos de forma incremental, probándose por regresión que los módulos ya integrados previamente siguieran funcionando de forma óptima.

Una vez logrado el prototipo integrador con todas las funcionalidades de la versión v1.0 se prosiguió a la expansión en hardware del mismo para crear la versión v2.0 extrayendo los módulos de joystick y display -que serían posteriormente agregados al sistema embebido del joystick- e incorporando los módulos de conectividad UDP sobre WiFi.

Tras lograr la versión v2.0 se repite el proceso de control de calidad de los diferentes módulos ya integrados.

A continuación se detallan las diferentes pruebas realizadas

\section{Verificación técnica de los diferentes módulos}


Todos los módulos fueron probados mediante una inspección visual durante el proceso de pruebas de concepto.


\subsection{Verificación del módulo de joystick}
Se verificó visualmente que los valores del joystick analógico puedan ser leídos apropiadamente, y que sean representativos y relevantes con la dirección del movimiento de la palanca sobre sus coordenadas X e Y.

\subsection{Verificación del módulo de control del display}
Se verificó visualmente que el display representa los caracteres programados en la prueba de concepto con una intensidad de luz aceptable para poder leerlos apropiadamente.


\subsection{Verificación del módulo de control de motores}
Se verificó visualmente que individualmente el motor pudiera girar en ambos sentidos. Luego, al implementarse los cuatro motores con sus ruedas, se probó que se pueda realizar los giros en todas las direcciones.

\subsection{Verificación del módulo de medición de temperatura y humedad}
Se verificó visualmente que los valores obtenidos por el sensor DHT11 fueran cercanos a lo esperado en relación a la temperatura en el interior del lugar de experimentación y la humedad en un valor cercano a lo reportado en Google.

\subsection{Verificación del modulo de medicion de presion atmosferica}
Se verificó visualmente que el valor obtenido por el sensor BMP280 fuera cercano a lo esperado en relación al valor reportado por Google.


\subsection{Verificación del módulo de medición de luminosidad}
Se verificó visualmente que los valores obtenidos del fotorresistor, tras ser transformados a valores absolutos porcentuales, guardan relación con el nivel de luminosidad ambiental del interior del lugar de experimentación.


\subsection{Verificación del módulo de comunicación UTP sobre WiFi}
Por medio de un dos programas UDP, uno cliente y uno servidor, se probó el establecimiento de la comunicación UDP entre dos ESP32. Luego se incorporó el servicio de comunicaciones UDP en el robot y mientras que desde el programa cliente se enviaban las acciones representando las direcciones del movimiento a realizar (FORWARD, BACKWARD, LEFT, RIGHT) y se observó visualmente como el robot giraba sus ruedas en función de los comandos enviados. Finalmente se incorporó el módulo de comunicaciones en el joystick para enviar los comandos al robot cuando se realizaba dicho


\section{Pruebas funcionales del producto}


\subsection{Prueba del módulo de medición de temperatura y humedad}

Se compararon los valores medidos por el módulo de medición de temperatura y humedad basado en el sensor DHT11 con los obtenidos a través de un dispositivo de medición de temperatura y humedad. Se realizó la medición en diferentes contextos:

\begin{itemize}
	\item En el interior de una vivienda.
	\item En el exterior durante el día.
	\item En el exterior durante la noche.
\end{itemize}

En la siguiente tabla se pueden apreciar los resultados.

\begin{table}[h]
\centering
\caption[Resultados de mediciones de temperatura y humedad]{Resultados de mediciones de temperatura y humedad}
\begin{tabular}{l c c c c}
\toprule
\textbf{Contexto} & \textbf{Temp. Robot} & \textbf{Temp. Ref.} & \textbf{Hume. Robot}  & \textbf{Hume. Ref.}\\
\midrule
Interior & 22.0 & 23.6 & 44.0 - 45.0 & 58.5 \\
Exterior (dia) & 17.0  & 14.0 & 47.0 - 53.0 & 62.9 - 64.0 \\
Exterior (noche) & - & - & - & - \\
\bottomrule
\hline
\end{tabular}
\end{table}

En el siguiente video se puede apreciar el experimento \cite{Video_Prueba_TempHum}.

\subsection{Prueba del módulo de medición de presión}

Se compararon los valores medidos por el módulo de medición de presión basado en el sensor BMP280 con los obtenidos a través de un dispositivo manómetro digital. Se realizó la medición en el interior de la vivienda dos días distintos.

En la siguiente tabla pueden apreciarse los resultados obtenidos:

\begin{table}[h]
\centering
\caption[Resultados de mediciones de presión ambiental]{Resultados de mediciones de presión ambiental}
\begin{tabular}{l c c}
\toprule
\textbf{Contexto} & \textbf{Presión. Robot} & \textbf{Presión. Ref.} \\
\midrule
Dia 1 & 1013 & 1018.9 \\
Dia 2 & 1003.9 & 998 \\
\bottomrule
\hline
\end{tabular}
\end{table}

En el siguiente video se puede apreciar el experimento \cite{Video_Prueba_Presion}.

\subsection{Prueba del módulo de medición de luminosidad ambiental}

Se compararon los valores medidos por el módulo de medición de luminosidad basado en un fotoresistor percibidos por el ojo humano sin utilizar ningún dispositivo de medición. Se realizó la medición en diferentes escenarios

\begin{itemize}
	\item En interiores con luz concentrada sobre el sensor.
	\item En interiores con luz ambiental no concentrada.
	\item En interiores con a oscuras
	\item En el exterior de día.
	\item En el exterior de noche.
\end{itemize}

Los resultados mostraron que los valores porcentuales indicados por el módulo de medición de luminosidad son consistentes con los niveles de luz detectados por el ojo humano. En el siguiente video se puede apreciar el experimento \cite{Video_Prueba_Luminosidad}.

\subsection{Prueba del control y desplazamiento del robot}

Se verificó el control del desplazamiento del robot de forma visual por medio de accionar el joystick en las diferentes coordenadas (X;Y) y se controló que:

\begin{itemize}
	\item la dirección del movimiento del robot sea acorde al accionamiento del joystick
	\item el tiempo de respuesta en el movimiento del robot y tras accionar del joystick sea mínimo, permitiendo una buena experiencia de usuario
\end{itemize}

En el siguiente video puede apreciarse dicho experimento \cite{Video_Prueba_Movimiento}.


\subsection{Prueba del módulo de visualización de display}

Se verificó el funcionamiento del display visualizando las lecturas de los valores censados y transmitidos por el robot. Se controló que:

\begin{itemize}
	\item las lecturas sean nítidas y entendibles
	\item las unidades de medida están presentes
	\item haya un detalle de lo que se está midiendo acompañando las lecturas y la unidad de medida
	\item el nivel de luminosidad sea óptimo para permitir la lectura independientemente de la iluminación ambiental
	\item se presentan las lecturas de todos los valores observados
\end{itemize}

En el siguiente video se puede apreciar el experimento \cite{Video_Prueba_Visualizacion_Display}.

\section{Reportes de testing}

...


\section{Documentación del producto }

Se desarrolló la documentación del producto compuesta de los siguientes entregables
\begin{itemize}
	\item Documentación técnica \cite{Robot_Tecnical_doc}.
	\item Manual de usuario \cite{Robot_User_manual}.
\end{itemize}








