% Chapter Template

\chapter{Ensayos y resultados} % Main chapter title

\label{Chapter4} % Change X to a consecutive number; for referencing this chapter elsewhere, use \ref{ChapterX}

%----------------------------------------------------------------------------------------
%	SECTION 1
%----------------------------------------------------------------------------------------
Esta sección presenta los diferentes prototipos realizados para determinar la viabilidad de cada una de las funcionalidades provistas, la metodología de desarrollo, testing, y finalmente los entregables finales del trabajo.

\section{Proceso de desarrollo y aseguramiento de calidad}
\label{sec:pruebasHW}

Para el proceso de desarrollo se realizaron pruebas de concepto de las diferentes funcionalidades utilizando como materiales la bibligrafia encontrada en internet, las hojas de datos y los ejemplos de codigo provistos por el SDK y librerias empleadas. Una vez logrado el objetivo funcional de componente se optimizo y encapsulo cada modulo para ser integrado de manera individual a un prototipo integrador sin afectar el funcionamiento de cualquier otro modulo.
De esta manera se desarrollo un prototipo integrador como la sumatoria de todos los modulos de forma incremental, probandose por regresion que los modulos ya integrados previamente siquieran funcionando de forma optima.

Una vez logrado el prototipo integrador con todas las funcionalides de la version v1.0 se prosiguio a la expansion en hardware del mismo para crear la version v2.0 extrayendo los modulos de joystick y display -que serian posteriormente agregados al sistema embebido del joystick- e incorporando los modulos de conectividad UDP sobre WiFi.

Tras lograr la version v2.0 se repitio el proceso de control de calidad de los diferentes modulos ya integrados.

A continuacion se detallan las diferentes pruebas realizadas 

\section{Verificacion tecnica de los diferentes modulos}


Todos los modulos fueron probados mediante una inspeccion visual durante el proceso de pruebas de concepto. 


\subsection{Verificacion del modulo de joystick}
Se verifico visualmente que los valores del joystick analogico puedan ser leidos apropiadamente, y que sean representativos y relevantes con la direccion del movimiento de la palanca sobre sus coordenadas X e Y.

\subsection{Verificacion del modulo de control del display}
Se verifico visualmente que el display represente los caracteres programados en la prueba de concepto con una intensidad de luz aceptable para poder leerlos apropiadamente.


\subsection{Verificacion del modulo de control de motores}
Se verifico visualmente que individualmente el motor pudiera girar en ambos sentidos. Luego, al implementarse los cuatro motores con sus ruedas, se probo que se pueda realizar los giros en todas las direcciones.

\subsection{Verificacion del modulo de medicion de temperatura y humedad}
Se verifico visualmente que los valores obtenidos por el sensor DHT11 fueran cercanos a lo esperado en relacion a la temperatura en el interior del lugar de experimentacion y la humedad en un valor cercano a lo reportado en Google.

\subsection{Verificacion del modulo de medicion de presion atmosferica}
Se verifico visualmente que el valor obtenido por el sensor BMP280 fuera cercano a lo esperado en relacion al valor reportado por Google.


\subsection{Verificacion del modulo de medicion de luminosidad}
Se verifico visualmente que los valores obtenidos del fotorresistor, tras ser transformados a valores absolutos porcentuales, guarden relacion con el nivel de luminosidad ambiental del interior de lugar de experimentacion.


\subsection{Verificacion del modulo de comunicacion UTP sobre WiFi}
Por medio de un dos programas UDP, uno cliente y uno servidor, se probo el establecimiento de la comunicacion UDP entre dos ESP32. Luego se incorporo el servicio de comunicaciones UDP en el robot y mientras que desde el programa cliente se enviaban las acciones representando las direcciones del movimiento a realizar (FORWARD, BACKWARD, LEFT, RIGHT) y se observo visualmente como el robot giraba sus ruedas en funcion de los comandos enviados. Finalmente se incorporo el modulo de comunicaciones en el joystick para enviar los comandos al robot cuando se realizaba dicho


\section{Pruebas funcionales del producto}


\subsection{Prueba del modulo de medición de temperatura y humedad}

Se compararon los valores medidos por el modulo de medicion de temperatura y humedad basado en el sendor DHT11 con los obtenidos a traves de un dispositivo de medicion de temperatura y humedad. Se realizo la medicion en diferentes contextos:

\begin{itemize}
	\item En el interior de una vivienda.
	\item En el exterior durante el dia.
	\item En el exterior durante la noche.
\end{itemize}

En la siguiente tabla se pueden apreciar los resultados.

\begin{table}[h]
\centering
\caption[Resultados de mediciones de temperatura y humedad]{Resultados de mediciones de temperatura y humedad}
\begin{tabular}{l c c c c}
\toprule
\textbf{Contexto} & \textbf{Temp. Robot} & \textbf{Temp. Ref.} & \textbf{Hume. Robot}  & \textbf{Hume. Ref.}\\
\midrule
Interior & 22.0 & 23.6 & 44.0 - 45.0 & 58.5 \\
Exterior (dia) & 17.0  & 14.0 & 47.0 - 53.0 & 62.9 - 64.0 \\
Exterior (noche) & - & - & - & - \\
\bottomrule
\hline
\end{tabular}
\end{table}

En el siguiente video se puede apreciar el experimento \cite{Video_Prueba_TempHum}.

\subsection{Prueba del modulo de medición de presión}

Se compararon los valores medidos por el modulo de medicion de presion basado en el sendor BMP280 con los obtenidos a traves de un dispositivo marometro digital. Se realizo la medicion en el interior de la vivienda dos dias distintos.

En la siguiente tabla pueden apreciarse los resultados obtenidos:

\begin{table}[h]
\centering
\caption[Resultados de mediciones de presion ambiental]{Resultados de mediciones de presion ambiental}
\begin{tabular}{l c c}
\toprule
\textbf{Contexto} & \textbf{Presion. Robot} & \textbf{Presion. Ref.} \\
\midrule
Dia 1 & 1013 & 1018.9 \\
Dia 2 & 1003.9 & 998 \\
\bottomrule
\hline
\end{tabular}
\end{table}

En el siguiente video se puede apreciar el experimento \cite{Video_Prueba_Presion}.

\subsection{Prueba del modulo de medición de luminosidad ambiental}

Se compararon los valores medidos por el modulo de medicion de luminosidad basado en un fotoresistor percibidos por el ojo humano sin utilizar ningun dispositivo de medicion. Se realizo la medicion en diferentes escenarios

\begin{itemize}
	\item En interiores con luz concentrada sobre el sensor.
	\item En interiores con luz ambiental no concentrada.
	\item En interiores con a oscuras
	\item En el exterior de dia.
	\item En el exterior de noche.
\end{itemize}

Los resultados mostraron que los valores porcentuales indicados por el modulo de medicion de luminosidad son consistentes con los niveles de luz detectados por el ojo humano. En el siguiente video se puede apreciar el experimento \cite{Video_Prueba_Luminosidad}.

\subsection{Prueba del control y desplazamiento del robot}

Se verifico el control del desplazamiento del robot de forma visual por medio de accionar el joystick en las diferentes coordenadas (X;Y) y se controlo que:

\begin{itemize}
	\item la direccion del movimiento del robot sea acorde al accionamiento del joystick
	\item el tiempo de respuesta en el movimiento del robot y tras accionar del joystick sea minimo, permitiendo una buena experiencia de usuario
\end{itemize}

En el siguiente video puede apreciarse dicho experimento \cite{Video_Prueba_Movimiento}. 


\subsection{Prueba del modulo de visualización de display}

Se verifico el funcionamiento del display visualizando las lecturas de los valores sensados y transmitidos por el robot. Se controlo que: 

\begin{itemize}
	\item las lecturas sean nitidas y entendibles
	\item las unidades de medida esten presentes
	\item haya un detalle de lo que se esta midiendo acompaniando las lecturas y la unidad de medida
	\item el nivel de luminosidad sea optimo para permitir la lectura independientemente de la iluminacion ambiental
	\item se presenten las lecturas de todos los valores observados 
\end{itemize}

En el siguiente video se puede apreciar el experimento \cite{Video_Prueba_Visualizacion_Display}.

\section{Reportes de testing}

...


\section{Documentacion del producto }

Se desarrollo la documentacion del producto compuesta de los siguientes entregables
\begin{itemize}
	\item Documentacion tecnica \cite{Robot_Tecnical_doc}.
	\item Manual de usuario \cite{Robot_User_manual}.
\end{itemize}




